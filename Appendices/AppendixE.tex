\chapter{Short historical overview}
\label{sec:history}
Soon after systematically experimenting with generating what will end up being known in the English languages as \textit{X-rays}, the German physicist Wilhelm R{\"o}ntgen used this new form of high energy radiation to ``take a picture'' of his wife’s hand. To everyone’s amazement the picture taken in 1895 showed the bones of her hand wearing her wedding ring. While the medical applications of X-rays are impressive in their own right, a material scientist will claim that the best application of X-rays was yet to come.

Not even twenty years later, another German physicist, Max von Laue, decided to use this new radiation, which he'd probably called R{\"o}ntgen rays,  to ``take a picture'' of a crystal. He actually had a good reason to try that. Von Laue expected the wavelength of these radiations to be of the same order of magnitude as the distance between atoms in matter. If this criterion is met then the wave behaviour of the radiation will suffer constructive interference along directions dictated by the crystal lattice planes. What he observed in his image was a series of ordered spots which would end up telling a story about the particular ordered arrangement of the crystal atoms. From here on things move rather quickly. Table~\ref{table:historyDiff} shows a quick overview over about a hundred years of historical events building to the development of electron diffraction techniques.

It is interesting to notice the interplay between experimental observations and theoretical predictions in the development of diffraction as a field. The observation of X-ray diffraction supported, first of all, the controversial wave-like behaviour of particles narrative, and, second of all, the description of crystalline materials as periodic lattice structures. It also led to the development of Bragg's law. Moreover, since the theory of space group symmetry had already been developed by the Russian crystallographer E. S. Fedorov, X-ray diffraction quickly became the structure analysis tool of choice of crystallography.








  

\renewcommand{\arraystretch}{1.3}

\begin{table}[htpb]
\caption{Partial chronology of the history of diffraction. }
\label{table:historyDiff}
\centering
\begin{tabular}{p{1.2cm}p{11cm}}
\toprule
\tabhead{Year} & \tabhead{Event} \\
\midrule

  {\small 1912} & {\small First X-ray diffraction experiments by W. Friedrich, P. Knipping and M.~von Laue( i.}  \\
  
  {\small 1913} & {\small Bragg family derives their name bearing law~\cite{Bragg13} to describe the geometry of diffraction spots.}\\
  
  {\small 1914} & {\small C. G. Darwin~\cite{Darwin14} derives the first dynamical theory for the intensities of the diffraction spots.}\\
   
  {\small 1924} & {\small L. de Broglie hypothesises that particles should also behave as waves.}\\
  
  {\small 1927} & {\small Independently, G. P. Thomson~\cite{Thomson27} at the University of Aberdeen and C. J. Davisson and L. H Germer~\cite{Davisson27} at the Bell Labs, observe low energy electron diffraction spots through thin films.}\\
     
  {\small 1928} & {\small H. Bethe~\cite{Bethe28} uses eigenvalue equations to explain and predict intensities in electron diffraction images.}\\
      
  {\small 1928} & {\small First electron diffraction patterns are recorded by S. Nishikawa and S.~Kikuchi~\cite{Nishikawa28} from grazing incidence geometry and are described as ``black and white lines in pairs due to multiple scattering and selective reflection''.}\\

  {\small 1931-4} & {\small E. Ruska and M. Knoll build the first electron microscope (EM), later known as a direct or transmission EM (TEM). }\\
  
  {\small 1935-8} & {\small M. Knoll suggests the idea of a scanning EM (SEM). M. von Ardenne~\cite{Ardenne38} constructs the first one.}\\
  
  {\small 1961} & {\small A. Howie and M. J. Whelan~\cite{Howie61} expands Darwin's theory to develop a simultaneous differential equation form for electron diffraction applicable for predicting diffraction contrast.}\\
  
  {\small 1960-5} & {\small P. B. Hirsch and co-workers at Cambridge~\cite{Hirsch60, electronMicroscopy} develop the theory of electron diffraction contrast which can be used to identify line and planar defects in thin films in TEM images.}\\
 
  {\small 1967} & {\small D. G. Coates~\cite{Coates67} observes electron diffraction patterns in the SEM later labelled electron channelling patterns (\textbf{ECPs}). }\\
  
  {\small 1967} & {\small  G. R.~Booker~\cite{Booker67} provides a theoretical interpretation for the phenomena based on the Darwin-Howie-Whelan theory. He also notices that the backscattered intensity varies rapidly with orientation close to Bragg condition, which will develop in a new technique of its own known as electron channelling contrast imaging (\textbf{ECCI}).}\\

    {\small 1973} & {\small J. A. Venables and C. J. Harland~\cite{Venables73} describe another diffraction technique capable of providing local crystallographic information termed electron backscattering patterns (\textbf{EBSP}) or latter electron backscattered diffraction (\textbf{EBSD}). }\\
  
   {\small 2012} & {\small The latest diffraction technique in the SEM with improved spatial resolution compared to EBSD is reported as transmission EBSD (\textbf{t-EBSD}) by R. R. Keller and R. H. Geiss~\cite{Keller12} and Transmission Kikuchi Diffraction (\textbf{TKD}) by P. W. Trimby~\cite{Trimby12}. }\\
\bottomrule
\end{tabular}
\end{table}
\renewcommand{\arraystretch}{1.}

While Bragg's law was good enough to explain the geometry of the diffraction spots, it proved inconsistent in predicting experimental X-ray diffraction intensities. This lead C. G. Darwin to develop a first form of dynamical diffraction theory. He took into account the interaction of X-rays with matter as partially transmitted and partially reflected amplitudes at each lattice plane. His theory predicted correct values for the reflected intensities. Later, in 1917, P. P. Ewald introduced a new form of dynamical theory in which he considered the crystal to be a periodic distribution of dipoles excited by the incident wave. The new theory predicted both transmitted and reflected intensities. 

There were still limitations in the dynamical theories. In 1930, J. A. Prins~\cite{Prins30} modified Darwin's theory to take into account the fact that the crystal is an absorbing medium. Just a year later, in 1931, von Laue showed that the interaction in Ewald's theory can be described by solving Maxwell's equations in a continuous medium with dielectric susceptibility distributed periodically in three dimensions. It is in this form that the dynamical diffraction theory is most used today and also the one we will use in this work.
  
The inverse effect was triggered by de Broglie's doctoral thesis in which he speculates that all particles ought to also behave as waves. The discovery of the remarkable phenomena that is electron diffraction provided the experimental means for the development of quantum mechanics. It soon became apparent that the diffraction theory must again be expanded to account for the more complex electron interaction with matter. Indeed, a number of dynamical theories for electrons have been developed over the years all carrying the legacy of X-ray diffraction theoretical interpretations.   

Electrons diffracting through crystals gives access to a plethora of information. When, in 1928, Shoji Nishikawa and Seishi Kikuchi directed a beam of \SI{50}{\kilo \electronvolt} electrons on a calcite sample at a grazing incidence of \SI{6}{\degree} diffraction was a phenomena associated with spots. But they have seen ``... black and white lines in pairs due to multiple scattering and selective reflection'' which we now recognise as diffraction patterns. A number of studies quickly followed the investigation of this new band features in electron diffraction. Notable here is that in 1937 Boersch~\cite{Boersch37}(paper in German) already proposed that these observations could be explain using von Laue's dynamical theory of electron diffraction.


In 1948, a number of models have already been proposed to explain the full range of features in the Kikuchi patterns. Artmann~\cite{Artmann48} (paper in German) used the reciprocity law and solved the Schr{\"o}dinger equation for bound electrons in the three dimensional crystal potential to predict with good accuracy the intensity profile and geometry of Kikuchi bands. 

While a beautiful physics experiment, electron diffraction would have lacked practical application before the development of the electron microscope; first the direct mode in 1934, where electrons penetrated through a sample and the image was collected on the other side (TEM) and, a few years later, the scanning mode, in which the incident electron beam scans over the sample in a raster manner and the backscattered electrons are recorded one ``pixel'' at a time to form an image (SEM). 

More insights could be derived from the electron signal. The diffraction spots observed in the TEM or when at a grazing incidence angle in the SEM (both conditions in which the crystal volume with which the electron beam interacts is small enough to approximate the behaviour to kinematical diffraction) were identified to be related to the crystal structure. The electron channelling patterns (ECP) observed now in the SEM can provide  information on the bulk crystal structure and orientation. If the detector is placed in the SEM such that mostly forward scattered electrons are collected, then one can obtained Kikuchi lines (EBSD) from a very small crystal volume or, in the language of microscopy, high spatial resolution.


The timely development of models and computer-aided indexing solutions for patterns from all seven crystal systems meant that fully automated EBSD could replace X-ray pole figure measurements for texture analysis avoiding the limitation that come with X-ray analysis where the sample is tilted through large angles. The model developed by Howie and Whelan for electron diffraction intensity predictions in the TEM could be easily be modified and implemented on a computer, as Hirsch had shown, to account for dislocations displacement fields and therefore predict dislocation contrast in transmission mode open the world of defect characterisation. 


Even for far from perfect crystals, electron diffraction can provide great insight. By scanning over a micro-granular crystal in the SEM and recording the EBSD image for each pixel, one can pick up the orientation of individual grains using orientation indexing techniques. In the world of material science this is extremely powerful as a tool for mapping the quality of a new material in terms of its grains.  We will talk more in Chapter~\ref{chap:TKD} about the novel Kikuchi diffraction technique, known as TKD, that can increase the spatial resolution even further allowing the study of truly nano-structural materials.

By the late 80s and early 90s, with the development of computers,  we start talking about  pattern indexing software systems. A number of companies have focused on developing modern EBSD systems packaged with indexing software appeared and experienced reasonable success catering to the industry's requirements. Unfortunately, it also marked the steady decline of academic interest in open electron diffraction software. In chapter~\ref{chap:TKD} I describe one such unicorn, and how the TKD modality was implemented and what the models are predicting. 


In about the same time when EBSD was becoming popular in the world of material science, a new SEM diffraction technique was also gaining traction. If the geometry was such that the incident beam was close to a Bragg condition, high contrast around small crystallographic defects could be observed in the recorded images, similar to the case in TEM. This technique which will be known as electron channelling contrast imaging (ECCI). Since Booker already hinted at theoretical interpretation of these effects in relationship with the available Darwin-Howie-Whelan model, the next obvious step is  to extend and implement the dynamical model such that it takes into account such small phase perturbations and can predict contrast profiles for dislocations observed in SEM. This is exactly what I endeavour to do in  Chapter~\ref{chap:ECCI}. 