\chapter{Summary, discussion and further work}
\label{chap:Conclusion}


I believe I have learned a good deal of facts and skills during my PhD programme. I think that because I'm somewhat confident that re-deriving the results shown here with the knowledge of today, would only take me a fraction of the four years period; possible famous last words. Having therefore great value for me, I wanted to have this knowledge written down in as much detail as this format would afford. The result is the textbook style of the first two chapters. 

My experience from the inside of the ivory tower that is academia is that there is no such thing as a unified consciousness of science, if a fact is understood in a certain field, the knowledge doesn't automatically carry in another, and sometimes the same science must be re-learned fifty years later because there was no bridge (research funds) to carry the wisdom; defying altogether the naive perception of science I gained as an undergrad. I dedicated multiple pages in this document, then, not to ground-breaking new science, but to knowledge that ought to be more within reach to the average scientist dedicated to diffraction in the SEM. I also dedicated more time than I would like to admit to describing the crystallography of the wurtzite system, including a full appendix of point group theory. The later was admittedly mostly for myself, but it would be nice to see more often the correct crystallographic unit cell of wurtzite (and not the hcp one) in talks about diffraction. 

In Chapter~\ref{Chap:Diffraction} I covered  everything electron diffraction, including describing where the values of important diffraction parameters, like the structure factor, come from.  I also took it upon myself to show graphs for scattering factors for group-III nitride systems (AlN, GaN and InN) and explain the insights they provide. Not only intuitive ones like the fact that the atoms in AlN will elastically scatter fewer electrons and, therefore, give a poorer signal to noise when it comes to ECC imaging, but also less intuitive aspects, such as families of planes that are more densely packed (such as the \textit{a}-plane) scatter fewer electrons then their less packed counterparts (such as \textit{c}-plane). This is somewhat unfortunate for the study of \hkl[001] wurtzite nitrides in the forward scatter geometry in the SEM. 

I then go on to talk about the wurtzite structure factor in these systems and comment on their predictions, including the failure of Friedel's law for non-centrosymmetric systems and the systematic absences we can expect. Embarked with all these information I spend the rest of the chapter applying the Howie-Whelan dynamical model to the ECCI geometry. I make the argument the equations should still hold even if we replace the dependence on depth in the sample of the Bloch waves with that on the distance travelled by the primary beam.  

Having the two beam dynamical equations laid out, in Chapter~\ref{chap:ECCI} I go over adding the displacement field introduced by a threading dislocation and using this model to predict the observed dislocation associated contrast in ECCI which I call \textit{ECC-strain}. I spend some effort emphasising the importance of setting out the correct reference frame transformations. In the  end, I argue, the contrast profile of a dislocation observed in ECCI is nothing more than a map of the strain projection selected by the diffraction conditions. Having worked out the complex relationship between the many frames involved, looking at the physics predicted by the strain profile of dislocations becomes a piece of cake.  

For instance, I show how the sample tilt affects dramatically the surface relaxation of the ECC-strain and that, in turn, will enhance the contrast. I then interrogate the ECC-strain for forward model geometry about whether the diffraction condition or the Burger vector dominates the contrast profile. I conclude that edge TDs ECCI contrast profile generally follows the Burger vector while the diffraction condition mostly affects its magnitude. I also compare these predictions with   experimental data to confirm the behaviour.


All this analysis was done for a two beam dynamical model. This approach is a valid prediction of a strong two beam diffraction condition. This, in turn, is a likely condition to achieve for SEM electron energies which describe a smaller Ewald sphere, which will likely intersect a few reciprocal lattice points, as opposed to the large sphere the TEM electron energies correspond to. It would be interesting in future work, as already suggested by Prof. De Graef, to compare the two beam model prediction to the multi-beam one and assess how correct these assumptions are. 

For a fact, I suspect that the strong beam is not always fulfilled in ECC images. It is my inkling that, in fact, ECC images showing only bright intensity at dislocation positions, mostly in metals, \eg steel~\cite{Gutierrez09} or Al~\cite{Barnoush10}, are obtained closer to weak beam conditions where the deviation from the Bragg angle, $\mathbf{s}_g$ is large. In this case, instead of bright-dark contrast, the dislocation moves locally the reciprocal lattice point closer to the Ewald sphere, showing as higher diffracted intensity on the micrograph. This is an interesting case especially because the Laue geometry approximation will become less feasible. To my knowledge, literature is yet to address the physics and possible models of predictions for, what is a common representation of ECCI, high intensity at dislocation positions in the SEM. 

The fact of the matter is that ECCI models are nowhere near mature and here are two more points that I can think of where they are lacking. In terms of the TD displacement predictions, I approximated the anisotropy of wurtzite system to an isotropic one plus corrections. It would be more accurate to consider full anisotropy and models for that exist~\cite{Barnett71}. But this would make a small difference  compared to introducing grains in the continuum model~\cite{Read50}. Dislocations on grain boundaries are ought to introduce a very different displacement field that is incomparable to that introduced by dislocations in a true continuous medium~\cite{Van02}. And we know plenty of dislocations lie on the grain boundaries. Grain boundaries facilitate dislocation pile-ups and influence directly the hardness of a material. Since TEM requires sample to be as thin as the size scale of grains, the sample preparation will introduce relaxation that will affect the grain structure. ECCI is therefore the only possible non-invasive imaging technique of dislocations at boundaries. Unfortunately, due to the limited number of  investigations we don't yet know how to use it precisely.

The other point of improvement is related to the Monte Carlo discussion in Chapter~\ref{chap:TKD}. I approximate the incident coherent beam penetration depth using Monte Carlo models that do not take into account diffraction whatsoever. I use the escape depth of low loss electrons, but the model does not know that diffracting electrons suffer less inelastic scattering and can travel further. Since the values I obtain are on the same scale as the literature offers, I think the values should be in the right ballpark, but it would be a worthwhile effort to quantitative study of how deep in the sample the ECC signal comes from.

Chapter~\ref{chap:TKD} discusses the importance of taking into account multiple electron energies when modelling EBSD/TKD patterns. It shows that for the TKD geometry, the low side of the energy distribution of the low loss electrons contributing to the Kikuchi patterns will be highly anisotropic on the detector. In other words, not taking into account an energy distribution for the low loss electrons will fail to predict not only the intensity distribution along the Kikuchi lines, but also the their width variation as well as details.

We also show how the sample thickness acts in effect as an energy filtering mechanism for the diffracting electrons in TKD. In terms of comparing the Kikuchi patterns in all the diffraction mechanism in the SEM discussed in this Thesis, Fig.~\ref{fig:MPs} on page~\pageref{fig:MPs} is an insightful one. Comparing the energy distribution of electrons contributing to ECP, TKD and EBSD patterns, and with it the resulting sharpness and details of the simulated Kikuchi lines on the MP, we conclude that the TKD modality is bridging the gap between the very narrow energy window of the ECPs and the broader energy range of EBSD. 


\section{Epilogue -- Science as an incremental, open process}

The history of science is all too often taught as a chronological list of discoveries. There is undeniable value in this approach as it reflects the arrow of complexity of notions. However, it leads to a very simplified image of the development of scientific knowledge: one big idea bringing over the next and so on. On page~\pageref{table:historyDiff}, I too show the history of diffraction as a table of chronological events. These are big shift events, which radically and permanently changed the way future science was to be done in this area. A good number of names in this table were awarded for their significant contribution with Nobel prizes. Nevertheless, the table is clearly a gross simplification of history, omitting, due to lack of space, the incremental refinement and maintenance work that supported and propelled the bigger ideas. It is quite common for important work of individual voices to be wiped away from science history as we associate a breakthrough to a single name or even to a small group of people. Seeing the bigger picture is, undeniably, worthwhile, but we must not mistake it for the full picture.

The ``unremarkable'' work done by the rest of the community, not awarded prestigious prizes, is not less important for the advancement of science. Quite the opposite. Neither science nor culture truly advance in big steps. In a recent study published in Nature, Miu \etal~\cite{Miu2018} looked at the way pieces of software get improved by a community of developers in a simulation of cumulative cultural evolution. One of the observations was that the vast majority of advances are of an incremental type and not, as the scientific community expect, leaps in knowledge. Observing the strong positive breakthrough bias of scientific publishing, one would find it hard to assume that enough credit is given to the ``tweakers''. 

Another critical observation was that big changes in the paradigm are more likely to turn out unsuccessful than smaller tweaks. Remember the Nobel prize in medicine awarded for the ``discovery'' of brain lobotomies\footnote{ ``for his discovery of the therapeutic value of leucotomy in certain psychoses''-- The Nobel Prize in Physiology or Medicine 1949~\cite{Nobel49}.}? Thankfully, neuroscience moved away from this particular scientific breakthrough. And it did that with small, incremental improvements on the understanding of the brain. Any sort of conversation about the development of science focused only on the leaps of knowledge must ultimately be misrepresenting the scientific process.


In this paradigm of scientific value misrepresentation, scientific code suffers perhaps even more. The philosopher Daniel C. Dennett, in his latest book \textit{From bacteria to Bach and back}~\cite{Dennett} makes the case that evolution is not only a good protocol for developing fit biological organisms but can, in fact, be successfully applied to a variety of concepts, perhaps, he argues, consciousness, the human mind and even code development. The latter analogy I find compelling. Similarly to adaptable organism having emerged from surviving a variety of conditions, the power of good code stands in the number of iterations it went through. Of course, we cannot wait around for functional code to ``occur'' as the results of tens of millions of years of iterations, and, after all, we expect developers to be somewhat wiser than the random processes occurring in nature. Nevertheless, in the end, each iterative step has the chance to rectify errors or limitations in the code, weed out unnecessary/old lines and replace them with new, more optimised, features. Established software tends to be software reviewed by many pairs of eyes. Yet, scientific software continues to be developed and  maintained by small groups and destined to see the light of only a handful of iterations.  


To add insult to injury, scientific code is rarely developed to be open and even more rarely made easily accessible. Here is another example of anecdotal evidence of why I think this a counter-intuitive way of following the scientific method. Two condensed matter groups set out, independently, to predict the behaviour of supercooled water, and, even though they implemented the same method, their results contradicted each other for seven straight years~\cite{supercool}. During this time, while the groups were in contact with one another, the actual lines of code never changed hands. When it finally did, a bug was discovered by the ``competing'' team in just a few months. I'm pointing out that we could have known in a few months, not seven long years, that water is predicted to change phase when supercooled. When scientific groups working in the same field do not collaborate with each other for whatever reason, it is science that suffers.


In the light of all these, I want my thesis work to make a positive tweak in the endeavour of making electron diffraction in the SEM a well-understood phenomena in the electron microscopy community. I aim for this work to aid the understanding of why we can observe and how we can study dislocations in the SEM and I do not expect it to be the definitive attempt. For these reasons I tried to make this document as accessible as possible for whoever wants to continue on this journey. I tried to explain in depth the building blocks I used and why I chose them, I provide access to whatever code I ran or wrote and I offer a small collection of extra materials. May your code and science be even a little bit better than mine!
%