\chapter{Summary and further work}
\label{chap:Conclusion}


I believe I have learned a good deal of facts and skills during my PhD programme. I think that because I'm somewhat confident that re-deriving the results shown here with the knowledge of today, would only take me a fraction of the four years period; possible famous last words. Having therefore great value for me, I wanted to have this knowledge written down in as much detail as this format would afford. The result is the textbook style of the first two chapters. 

My experience from the inside of the ivory tower that is academia is that there is no such thing as a unified consciousness of science, if a fact is understood in a certain field, the knowledge doesn't automatically carry in another, and sometimes the same science must be re-learned fifty years later because there was no bridge (research funds) to carry the wisdom; defying altogether the naive perception of science I gained as an undergrad. I dedicated multiple pages in this document, then, not to ground-breaking new science, but to knowledge that ought to be more within reach to the average scientist dedicated to diffraction in the SEM. I also dedicated more time than I would like to admit to describing the crystallography of the wurtzite system, including a full appendix of point group theory. The later was admittedly mostly for myself, but it would be nice to see more often the correct crystallographic unit cell of wurtzite (and not the hcp one) in talks about diffraction. 

In Chapter~\ref{Chap:Diffraction} I covered  everything electron diffraction including describing where the values of important diffraction parameters, like the structure factor, come from.  I also took it upon me to show graphs for scattering factors for group-III nitride systems (AlN, GaN and InN) and explain the insights they provide. Not only intuitive ones like the fact that the atoms in AlN will elastically scatter fewer electrons and, therefore, give a poorer signal to noise when it comes to ECC imaging, but also less intuitive aspects, such as families of planes that are more densely packed (such as the \textit{a}-plane) scatter fewer electrons then their less packed counterparts (such as \textit{c}-plane). This is somewhat unfortunate for the study of \hkl[001] wurtzite nitrides in the forward scatter geometry in the SEM. 

I then go on to talk about the wurtzite structure factor in these stems and comment on their predictions, including a chance to mention to failure of Friedel's law for non-centrosymmetric systems and the systematic absences we can expect. Embarked with all these information I spend the rest of the chapter applying the Howie-Whelan dynamical model to the ECCI geometry. I make the argument the equations should still hold even if we replace the dependence on depth in the sample of the Bloch waves with that on the distance travelled by the primary beam.  

Having the two beam dynamical equations laid out, in Chapter~\ref{chap:ECCI} I go over adding the displacement field introduced by a threading dislocation and using this model to predict the observed dislocation associated contrast in ECCI. We get sidetracked for a few pages before that, for me to make the case that channelling is a gross misnomer for diffraction. 

I spend some effort emphasising the importance of setting out the correct reference frame transformations. In the  end, I argue, the contrast profile of a dislocation observed in ECCI is nothing more that a map of the strain projection seleced by the diffraction conditions. These many frames involved, once 